
\documentclass{beamer} 


\mode<presentation>
{
  \usetheme{Berkeley}
  % or ...

  \setbeamercovered{transparent}
  % or whatever (possibly just delete it)
}

\usepackage{tikz}
\usepackage{graphicx}
\usepackage[english]{babel}
% or whatever
\usepackage{array}


\usepackage{relsize, lscape}
\usepackage{multirow, booktabs, adjustbox}

\newcolumntype{P}[1]{>{\raggedright\arraybackslash}p{#1}}






\usepackage{adjustbox}

\usepackage{listings, amsmath}
\usepackage{natbib}

\usepackage{hyperref}
\hypersetup{
    colorlinks,
    citecolor=blue,
    linkcolor=blue,
    urlcolor=blue,
    backref=true
}


\urlstyle{same}


\def\gray{\color{lightgray}}
\def\red{\color{red}}
\def\white{\color{white}}
\def\brown{\color{brown}}
\def\black{\color{black}}

\usepackage[utf8]{inputenc}
% or whatever

\usepackage{times}
%\usepackage[T1]{fontenc}
% Or whatever. Note that the encoding and the font should match. If T1
% does not look nice, try deleting the line with the fontenc.
\usepackage{tikz}
\usetikzlibrary{decorations.pathreplacing,angles,quotes}
\usetikzlibrary{shapes, shadows, arrows, positioning}




\title[] % (optional, use only with long paper titles)
{Reproducible Analytics at the World Bank}

\subtitle
{}

\author[] % (optional, use only with lots of authors)
{Garret~Christensen\inst{1}\inst{2}\\
Fernando~Hoces de la Guardia\inst{1}}
% - Give the names in the same order as the appear in the paper.
% - Use the \inst{?} command only if the authors have different
%   affiliation.

\institute[Universities of Somewhere and Elsewhere] % (optional, but mostly needed)
{
  \inst{1}%
  UC Berkeley:\\
  Berkeley Initiative for Transparency in the Social Sciences\\
  \inst{2}%
  Berkeley Institute for Data Sciences\\
}

% - Use the \inst command only if there are several affiliations.
% - Keep it simple, no one is interested in your street address.

\date[BITSS2017] % (optional, should be abbreviation of conference name)
{World Bank, January 2018}
%\\
%Slides available online at \url{https://goo.gl/RtfxqX}}
% - Either use conference name or its abbreviation.
% - Not really informative to the audience, more for people (including
%   yourself) who are reading the slides online

\subject{Research Transparency}
% This is only inserted into the PDF information catalog. Can be left
% out. 


\pgfdeclareimage[height=2cm]{university-logo}{../Images/BITSSlogo.png}
\logo{\pgfuseimage{university-logo}}

% If you have a file called "university-logo-filename.xxx", where xxx
% is a graphic format that can be processed by latex or pdflatex,
% resp., then you can add a logo as follows:

% \pgfdeclareimage[height=0.5cm]{university-logo}{university-logo-filename}
% \logo{\pgfuseimage{university-logo}}



% Delete this, if you do not want the table of contents to pop up at
% the beginning of each subsection:
%\AtBeginSubsection[]
%{
%  \begin{frame}<beamer>{Outline}
%    \tableofcontents[currentsection,currentsubsection]
%  \end{frame}
%}


% If you wish to uncover everything in a step-wise fashion, uncomment
% the following command: 

\beamerdefaultoverlayspecification{<.->}


\begin{document}

\begin{frame}
  \titlepage
\end{frame}




% Structuring a talk is a difficult task and the following structure
% may not be suitable. Here are some rules that apply for this
% solution: 

% - Exactly two or three sections (other than the summary).
% - At *most* three subsections per section.
% - Talk about 30s to 2min per frame. So there should be between about
%   15 and 30 frames, all told.

% - A conference audience is likely to know very little of what you
%   are going to talk about. So *simplify*!
% - In a 20min talk, getting the main ideas across is hard
%   enough. Leave out details, even if it means being less precise than
%   you think necessary.
% - If you omit details that are vital to the proof/implementation,
%   just say so once. Everybody will be happy with that.
%%%%%%%%%%%%%%%%%%%%%%%%%%%%%%%%%%%%%%%%%%%%%%%%%%%%%%%%%%%%%%%%%%%%%%%
%%%%%%%%%%%%%%%%%%%%%%%%%%%%%%%%%%%%%%%%%%%%%%%%%%%%%%%%%%%%%%%%%%%%%

%\begin{frame}{Outline}
%  \tableofcontents
  % You might wish to add the option [pausesections]
%\end{frame}


{ % all template changes are local to this group.
    \setbeamertemplate{navigation symbols}{}
    \begin{frame}[plain]
        \begin{tikzpicture}[remember picture,overlay]
            \node[at=(current page.center)] {
                \href{https://www.bitss.org/}{\includegraphics[width=\paperwidth]{../Images/bitsslogo.png}}
            };
        \end{tikzpicture}
     \end{frame}
}

\begin{frame}{Research Transparency}
\textbf{Issues:}
\begin{itemize}
\item Scientific misconduct
\item Publication Bias
\item Specification searching / P-Hacking
\item Replications problems
\end{itemize}
\textbf{Solutions:}
\begin{itemize}
\item Ethical research
\item Registrations
\item PAPs
\item Guidelines and Protocols
\end{itemize}
\medskip

\textbf{Want to learn more?:} MOOC, RT2 Amsterdam, and more! 
\end{frame}



\begin{frame}{New Dimension to Increase Transparency and Reproducibility: Policy Analysis}


\
\tikzstyle{agent} = [diamond, draw, node distance= 7em, minimum height=5em, minimum width=5em]
\tikzstyle{line} = [draw, -stealth]

\tikzstyle{line_enph} = [draw, red, ultra thick]
\tikzstyle{inp} = [draw, circle, text centered, minimum height=2em, text width=2em, node distance= 10em]
\tikzstyle{outp} = [draw, circle, text centered, minimum height=2em, text width=2em, node distance= 10em]
\tikzstyle{block1} = [draw, circle,  text width=5em, text centered, minimum height=7em, node distance= 5em]
\tikzstyle{block2} = [draw, circle,  text width=7em, text centered, minimum height=2em, node distance= 5em]
\tikzstyle{block3} = [draw, rounded rectangle,  text width=5em, text centered, minimum height=2em, node distance= 5em]


\begin{figure}[h!]\centering
\vspace{-2.0em}
\begin{tikzpicture}[thick,scale=0.55, every node/.style={scale=0.55}]

%% The oddly shaped truth
\node[regular polygon, regular polygon sides=3,
              draw, fill=white,
              inner sep=.1em,
              shape border rotate=70](tru){Truth};


%%%%%Researcher 1 and Inputs: depends on R_1%%%%%%%
\node [block1, right = 2em of tru](R_1){Research\linebreak $(R)$};
\node [block1,inner sep=0em, right = 2em of R_1](PA_1){Policy \linebreak Analysis: ($PA$) \linebreak  Gains  \& \linebreak losses};
\node [block1, above right = 0em and 8em of R_1](PM_1){Policy\linebreak Maker 1};
\node [block1, below right = 0em and 8em of R_1](PM_2){Policy\linebreak Maker 2};

\node [block3, right = 2em of PM_1](PC_1){Support};
\node [block3, right = 2em of PM_2](PC_2){Oppose};


%%%%% Draw edges col 2%%%%%%%%%%%%%%%%%
%Research  to PA
\draw [line](tru) -- (R_1);
\draw [line](R_1) -- (PA_1);
%PA to PM
\draw [line](PA_1) -- (PM_1);
\draw [line](PA_1) -- (PM_2);
%PM to PC
\draw [line](PM_1) -- (PC_1);
\draw [line](PM_2) -- (PC_2);

%%%%%%%%%%%%%%%%%%%%%%%%%%%%%%%%%%%%%%



\draw[decoration={brace}, decorate] (7,3.4) -- node[above=6pt](lab4){{\Large Observed by citizens} }(16,3.4);




\end{tikzpicture}
\vspace{-.8em}
\end{figure}
\end{frame}





\begin{frame}[shrink=30]{Credibility Crisis}



\tikzstyle{agent} = [diamond, draw, node distance= 7em, minimum height=5em, minimum width=5em]
\tikzstyle{line} = [draw, -stealth]

\tikzstyle{line_d} = [draw, dashed, -stealth]
\tikzstyle{inp} = [draw, circle, text centered, minimum height=2em, text width=2em, node distance= 10em]
\tikzstyle{outp} = [draw, circle, text centered, minimum height=2em, text width=2em, node distance= 10em]
\tikzstyle{block1} = [draw, circle,  text width=3em, text centered, minimum height=2em, node distance= 5em]
\tikzstyle{block2} = [draw, circle,  text width=4em, text centered, minimum height=2em, node distance= 5em]
\tikzstyle{block3} = [draw, rounded rectangle,  text width=5em, text centered, minimum height=2em, node distance= 5em]

\begin{figure}[h!]
\centering
\hspace*{0.2\linewidth}
\begin{tikzpicture}[thick,scale=0.2, every node/.style={scale=0.8}]

%% The oddly shaped truth
\node[regular polygon, regular polygon sides=3,
              draw, fill=white,
              inner sep=.1em,
              shape border rotate=70](tru){Truth};


%%%%%Researcher 1 and Inputs: depends on R_1%%%%%%%
\node [block1, right = 5em of tru](R_2){$R_2$};
\node [block1, above = 8em of R_2](R_1){$R_1$} node [left = -0.1em of R_1, text width = 5em]{Large Treatment Effect};
\node [block1, below = 8em of R_2](R_3){$R_3$} node [left = -0.1em of R_3, text width = 5em]{Small Treatment Effect};

\draw[decoration={brace,mirror}, decorate] (13,-33) -- node[below=6pt, text width = 4.8em](lab1){Researchers Degrees of 
Freedom}(18,-33);

\draw[decoration={brace,mirror}, decorate] (29,-33) -- node[below=6pt, text width = 4.8em](lab2){P. Analyst Degrees of 
Freedom}(34,-33);

\draw[decoration={brace}, decorate] (31,33) -- node[below=2pt]{Observed by citizens}(70,33);

%\node [agent](Researcher_1){$R_1$} node [left = 0.6em of Researcher_1, text width = 8em]{Large Treatment Effect};

\node [block1, right = 5em of R_1](PA_12){$PA_{1,2}$};
\node [block1, above = .5em of PA_12](PA_11){$PA_{1,1}$} node [right = -0.1em of PA_11, text width = 5em]{Large gains only};
\node [block1, below = .5em of PA_12](PA_13){$PA_{1,3}$};

\node [block1, right = 5em of R_2](PA_22){$PA_{22,}$};
\node [block1, above = .5em of PA_22](PA_21){$PA_{2,1}$};
\node [block1, below = .5em of PA_22](PA_23){$PA_{2,3}$};

\node [block1, right = 5em of R_3](PA_32){$PA_{3,2}$};
\node [block1, above = .5em of PA_32](PA_31){$PA_{3,1}$};
\node [block1, below = .5em of PA_32](PA_33){$PA_{3,3}$} node [right = -0.1em of PA_33, text width = 5em]{Large losses only};


\node [block2, above right = 3em and 15em of R_2](PM_1){Policy\linebreak Maker 1};
\node [block2, below right = 3em and 15em of R_2](PM_2){Policy\linebreak Maker 2};

\node [block3, right = 3em of PM_1](PC_1){Support};
\node [block3, right = 3em of PM_2](PC_2){Oppose};


%%%%% Draw edges col 2%%%%%%%%%%%%%%%%%
%Truth to research
\draw [line](tru) -- (R_1);
\draw [line_d](tru) -- (R_2);
\draw [line_d](tru) -- (R_3);


%Research  to PA
\draw [line_d](R_1) -- (PA_13);
\draw [line_d](R_1) -- (PA_11);
\draw [line](R_1) -- (PA_12);

\draw [line_d](R_2) -- (PA_23);
\draw [line_d](R_2) -- (PA_21);
\draw [line_d](R_2) -- (PA_22);

\draw [line_d](R_3) -- (PA_33);
\draw [line_d](R_3) -- (PA_31);
\draw [line_d](R_3) -- (PA_32);



%PA to PM
\draw [line](PA_11) -- (PM_1);
\draw [line](PA_33) -- (PM_2);
%PM to PC
\draw [line](PM_1) -- (PC_1);
\draw [line](PM_2) -- (PC_2);

%%%%%%%%%%%%%%%%%%%%%%%%%%%%%%%%%%%%%%




\end{tikzpicture}
\caption{Policy-making with low TR in research and policy analysis}\label{low_cred}
\end{figure}
%\end{comment}
\end{frame}

\begin{frame}{Relevance}

{
\large
\begin{itemize}
\item Policy makers can cherry-pick facts
\item Hinders automation and/or systematic improvements of reports
\item Unclear how research affects policy estimates
\end{itemize}
}
\end{frame}



%14 mins (5/6/3)
%\subsection[Parallels between Sci \& PA]{Parallels of T \& R between Science and Policy Analysis}
%5mins
\begin{frame}{Transparency and Reproducibility in Policy and Research}
 \centering
 \resizebox{\textwidth}{!}{%
\begin{tabular}{ P{3cm}  P{4cm}  P{5cm} }
     \toprule
     & Research & Policy Analysis \\
     \midrule \midrule
     Output 		 & Peer reviewed publication & Client-oriented policy report \\
     \hline
     Problems of low TR &  
     Ex: Publication Bias. 
     &  Ex:Low credibility. \\
     \hline
	 Common Solutions &  \multicolumn{2}{ P{9cm} }{ \centering
     Disclosure of key details. \linebreak Open data and materials. } \\
     \hline
     Common Tools &  \multicolumn{2}{ P{9cm} }{ \centering Dynamic documentation.  \linebreak Distributed version control. } \\
     \hline
     Specific Solutions & Ex:Test for reproducibility & Ex: Develop reproducibility\\
     \hline
     Who increases \linebreak TR & Researchers, Funders, Journals &   \textbf{Not} the policy analysts (Maybe: Policy schools, Think tanks, Media.  \textbf{Yes:} BITSS!) \\
     \toprule

\end{tabular}}
\end{frame}



%14 mins (5/6/3)
%\subsection[Parallels between Sci \& PA]{Parallels of T \& R between Science and Policy Analysis}
%5mins
\begin{frame}[noframenumbering]{Transparency and Reproducibility in Policy and Research}
 \centering
 \resizebox{\textwidth}{!}{%
\begin{tabular}{ P{3cm}  P{4cm}  P{5cm} }
     \toprule
     & Research & Policy Analysis \\
     \midrule \midrule
     Output 		 & Peer reviewed publication & Client-oriented policy report \\
     \hline
     Problems of low TR &  
     Ex: Publication Bias. 
     &  Ex:Low credibility. \\
     \hline
    \textbf{Common Solutions} &  \multicolumn{2}{ P{9cm} }{ \centering
  \textbf{   Disclosure of key details. \linebreak Open data and materials. } } \\
     \hline
     \textbf{Common Tools}  &  \multicolumn{2}{ P{9cm} }{ \centering \textbf{ Dynamic documentation.  \linebreak Distributed version control. } } \\
     \hline
     Specific Solutions & Ex:Test for reproducibility & Ex: Develop reproducibility\\
     \hline
     Who increases \linebreak TR & Researchers, Funders, Journals &   \textbf{Not} the policy analysts (Maybe: Policy schools, Think tanks, Media.  \textbf{Yes:} BITSS!) \\
     \toprule

\end{tabular}}
\end{frame}



\begin{frame}{Transparency and Reproducibility in Policy and Research }  

\begin{center}
{\LARGE
We don't know how the sausage is made!  

\pause
\bigskip
Let's follow science, open up the kitchen. And publish the cook book with the recipe. 
%Let’s follow open science and publish a cook book with the recipe
}
\end{center}

\end{frame}


\begin{frame}{Our Proposal for Open Policy Analysis}
\begin{enumerate}

\item Increase awareness (Motivational paper coming soon!)

\item Publish guidelines on transparency and reproducibility for policy analysis (similar to the TOP Guidelines for research). 

\item Partner with agencies/think tanks interested in implementing these ideas.
\texttt{\href{https://rpubs.com/fhoces/dd_cbo_mw}{\underline{Example here.}}}

\item Iterate.
\end{enumerate}
\end{frame}


\begin{frame}[noframenumbering]
\begin{center}
\vspace*{4em}
{\Large Thank you.\\
\bigskip
{\Large  \href{mailto:fhoces@berkeley.edu}{fhoces@berkeley.edu}  }
}
\end{center}
\end{frame}


\begin{frame}[noframenumbering]
\begin{center}
\vspace*{4em}
{\Large Backup Slides
}
\end{center}
\end{frame}



%6mins
\begin{frame}[shrink=25, label = pa_comp]{Guidelines Goal}

\tikzstyle{estimate} = [diamond, draw, node distance= 7em,text width = 5em, minimum height=5em, minimum width=5em, align = center]
\tikzstyle{line} = [draw, -stealth]
\tikzstyle{line_enph} = [draw, red, ultra thick]
\tikzstyle{inp} = [draw, rectangle, text centered, minimum height=3em, text width=2em, node distance= 2em]
\tikzstyle{source} = [draw, rectangle, text centered, minimum height=8em, text width=5em, node distance= 10em]
\tikzstyle{model} = [draw, rectangle, text centered, minimum height=8em, text width=15em, node distance= 10em]
%\tikzstyle{outp} = [draw, ellipse, text centered, minimum height=2mm, text width=2em, node distance= 10em]
%\tikzstyle{block} = [draw, rectangle,  text width=8em, text centered, minimum height=55mm, node distance= 5em]


%\begin{adjustbox}{max totalsize={1\textwidth}{.8\textheight},center}

\begin{figure}[h!]\centering 
\hspace*{-2.5em}
\begin{tikzpicture}[thick,scale=0.6, every node/.style={scale=0.6}]
\setbeamercovered{invisible}

%%%%%Nodes: Sources%%%%%%%
\onslide<1-5>\node [source](D_1){$Data$};
\onslide<1>\node [source, below = 1em of D_1](Lit){$Research$};
\onslide<1-5>\node [source, below = 1em of Lit](OR){\textit{Guess work}};


\draw[decoration={brace,mirror}, decorate] (-1.2,-10) -- node[below=6pt] {$Sources$}(1.1,-10);

\draw[decoration={brace,mirror}, decorate] (4.8,-11.3) -- node[below=6pt] {$Inputs$}(6.2,-11.3);


%node[below = 1em ](OR) -- node[below = 6em]{asd}(OR)


%\onslide<2-5>\node [source, color=red](Lit){$Research$};
\onslide<2-5>\node [source, below = 1em of D_1, color=red](Lit){$Research$};

%%%%%Nodes: Inputs%%%%%%%
\onslide<1-5>\node [inp, above right = 1em and 6em of  D_1 ](I_1){$I_1$};
\onslide<1-5>\node [inp, below = 1em of I_1](I_2){$I_2$};
\onslide<1-2>\node [inp, below = 8em of I_2](I_j){$I_j$};
\onslide<1-5>\node [inp, below right = 1em and 6em of OR ](I_last){$I_J$};

\onslide<3->\node [inp,  color=red, below = 8em of I_2](I_j){$I_j$};


\onslide<1-5>\path (I_2) -- node[auto=false, rotate=90, anchor=north, outer sep=-0.5em]{\ldots} (I_j);
\onslide<1-5>\path (I_j) -- node[auto=false, rotate=90, anchor=north, outer sep=-0.5em]{\ldots} (I_last);

%%%%%Paths connecting Sources with Inputs%%%%%%%
\onslide<1-5>\draw [line](D_1.east) -- (I_1.west);
\onslide<1-5>\draw [line](D_1.east) -- (I_2.west);
\onslide<1-2>\draw [line, opacity=1, anchor=center](Lit.east) -- (I_j.west);
\onslide<1-5>\draw [line, opacity=1, anchor=center](OR.east) -- (I_last.west);

\onslide<3->\draw [line,  color=red](Lit.east) -- (I_j.west);


\onslide<1-5>\draw [line, opacity=.3][xshift=1em](D_1.east) -- ([yshift=-8 em]I_1.west);
\onslide<1-5>\draw [line, opacity=.3][xshift=1em](D_1.east) -- ([yshift=-10 em]I_1.west);

\onslide<1-5>\draw [line, opacity=.3][xshift=1em](Lit.east) -- ([yshift=7 em]I_j.west);
\onslide<1-5>\draw [line, opacity=.3][xshift=1em](Lit.east) -- ([yshift=-5 em]I_j.west);

\onslide<1-5>\draw [line, opacity=.3][xshift=1em](OR.east) -- ([yshift=3 em]I_last.west);
\onslide<1-5>\draw [line, opacity=.3][xshift=1em](OR.east) -- ([yshift=5 em]I_last.west);


%%%%%Node and paths for model%%%%%%%
\onslide<1-3>\node [model, below right = 2em and 10em of D_1](model){$Model$};
\onslide<1-5>\draw [line](I_1) -| (model);
\onslide<1-5>\draw [line](I_2) -| (model);
\onslide<1-3>\draw [line](I_j) |- (model);
\onslide<1-5>\draw [line](I_last) -| (model);

\onslide<4->\node [model,  color=red, below right = 2em and 10em of D_1](model){$Model$};
\onslide<4->\draw [line,  color=red](I_j) |- (model);


%%%%%Node and paths for policy estimates%%%%%%%
\onslide<1-4>\node [estimate, right = 5em of model](PE_2){$Policy$ $Estimate_2$};
\onslide<1-4>\node [estimate, above = 5em of PE_2](PE_1){$Policy$ $Estimate_1$};
\onslide<1-4>\node [estimate, below = 5em of PE_2](PE_3){$Policy$ $Estimate_3$};


\onslide<1-4>\draw [line] (model.east) -- (PE_1.south west);
\onslide<1-4>\draw [line] (model.east) -- (PE_2.west);
\onslide<1-4>\draw [line] (model.east) -- (PE_3.north west);


\onslide<5->\node [estimate,  color=red, right = 5em of model](PE_2){$Policy$ $Estimate_2$};
\onslide<5->\node [estimate,  color=red, above = 5em of PE_2](PE_1){$Policy$ $Estimate_1$};
\onslide<5->\node [estimate,  color=red, below = 5em of PE_2](PE_3){$Policy$ $Estimate_3$};


\onslide<5->\draw [line,  color=red] (model.east) -- (PE_1.south west);
\onslide<5->\draw [line,  color=red] (model.east) -- (PE_2.west);
\onslide<5->\draw [line,  color=red] (model.east) -- (PE_3.north west);
\end{tikzpicture}
\end{figure}

\end{frame}





\begin{frame}{Summary of Adapted Guidelines}
 \centering
 \resizebox{25em}{10em}{%
    \begin{tabular}{ P{1.2cm} P{1.5cm} P{3cm}  P{3cm}  P{3cm} }
     \toprule
     {\black Standard} & {\black Level 0} 	& {\black Level 1} & {\black Level 2} & {\black Level 3} \\
     \midrule      \midrule
   {\black Workflow} & {\gray Policy estimates vaguely described}  & {\gray All the inputs, and their corresponding sources, used in the calculations are listed } & {\gray Lvl 1 + Policy estimates are listed, in same unit if possible} & {\gray Lvl 2 + all the components can be modified with little effort} \\
     \midrule
    {\black Data} & {\gray Report says nothing} & {\gray Clearly stated whether all, some components, or none of the data is available, with instructions for access when possible.} & {\gray Lvl 1 + report and data are in same place} & {\gray Lvl 2 + Report has specific lines of code that call the data and changes in the data produce traceable changes in the report} \\
     \midrule
      {\black Methods \& Code} & {\gray Key assumption are listed} & {\gray Methods are described in prose. Large amount of work is required to reproduce qualitatively similar estimates}  & {\gray Methods and described in prose, with detailed formulas, and code is provided as supplementary material} & {\gray Lvl 2 + All is in the same document where changes in the code affect the output automatically} \\
     \toprule
     \multicolumn{5}{ P{11.7cm} }{  \raggedleft {\black\small{ From TOP guidelines \citep{nosek2015promoting} v1.0.1}} }
   \end{tabular}}
\end{frame}


\begin{frame}[noframenumbering]{Summary of Adapted Guidelines}
 \centering
 \resizebox{25em}{10em}{%
    \begin{tabular}{ P{1.2cm} P{1.5cm} P{3cm}  P{3cm}  P{3cm} }
     \toprule
    {\black Standard} & {\red Level 0 } 	&  {\red Level 1 } &  {\red Level 2}  & {\red Level 3 }  \\
     \midrule      \midrule
    {\black Workflow} & {\gray Policy estimates vaguely described}  & {\gray All the inputs, and their corresponding sources, used in the calculations are listed } & {\gray Lvl 1 + Policy estimates are listed, in same unit if possible} & {\gray Lvl 2 + all the components can be modified with little effort} \\
     \midrule
    {\black Data} & {\gray Report says nothing} & {\gray Clearly stated whether all, some components, or none of the data is available, with instructions for access when possible.} & {\gray Lvl 1 + report and data are in same place} & {\gray Lvl 2 + Report has specific lines of code that call the data and changes in the data produce traceable changes in the report} \\
     \midrule
      {\black Methods \& Code} & {\gray Key assumption are listed} & {\gray Methods are described in prose. Large amount of work is required to reproduce qualitatively similar estimates}  & {\gray Methods and described in prose, with detailed formulas, and code is provided as supplementary material} & {\gray Lvl 2 + All is in the same document where changes in the code affect the output automatically} \\
     \toprule
     \multicolumn{5}{ P{11.7cm} }{  \raggedleft {\black\small{ From TOP guidelines \citep{nosek2015promoting} v1.0.1}}}
   \end{tabular}}
\end{frame}


\begin{frame}[noframenumbering]{Summary of Adapted Guidelines}
 \centering
 \resizebox{25em}{10em}{%
    \begin{tabular}{ P{1.2cm} P{1.5cm} P{3cm}  P{3cm}  P{3cm} }
     \toprule
     {\red Standard} & {\black Level 0} 	& {\black Level 1} & {\black Level 2} & {\black Level 3} \\
     \midrule      \midrule
    {\red Workflow} & {\gray Policy estimates vaguely described}  & {\gray All the inputs, and their corresponding sources, used in the calculations are listed} & {\gray Lvl 1 + Policy estimates are listed, in same unit if possible} & {\gray Lvl 2 + all the components can be modified with little effort} \\
     \midrule
    {\red Data} & {\gray Report says nothing} & {\gray Clearly stated whether all, some components, or none of the data is available, with instructions for access when possible.} & {\gray Lvl 1 + report and data are in same place} & {\gray Lvl 2 + Report has specific lines of code that call the data and changes in the data produce traceable changes in the report} \\
     \midrule
      {\red Methods \& Code} & {\gray Key assumption are listed} & {\gray Methods are described in prose. Large amount of work is required to reproduce qualitatively similar estimates}  & {\gray Methods and described in prose, with detailed formulas, and code is provided as supplementary material} & {\gray Lvl 2 + All is in the same document where changes in the code affect the output automatically} \\
     \toprule
     \multicolumn{5}{ P{11.7cm} }{  \raggedleft {\black\small{ From TOP guidelines \citep{nosek2015promoting} v1.0.1}} }
   \end{tabular}}
\end{frame}

\begin{frame}[noframenumbering, label=guidelines_sum]{Summary of Adapted Guidelines}
 \centering
 \resizebox{25em}{10em}{%
    \begin{tabular}{ P{1.2cm} P{1.5cm} P{3cm}  P{3cm}  P{3cm} }
     \toprule
     Standard & Level 0 	& Level 1 & Level 2 & Level 3 \\
     \midrule      \midrule
   {\black Workflow} & {\gray Policy estimates vaguely described}  & {\gray All the inputs, and their corresponding sources, used in the calculations are listed } & {\gray Lvl 1 + Policy estimates are listed, in same unit if possible} & {\gray Lvl 2 + all the components can be modified with little effort} \\
     \midrule
    {\black Data} & {\gray Report says nothing} & {\gray Clearly stated whether all, some components, or none of the data is available, with instructions for access when possible.} & {\gray Lvl 1 + report and data are in same place} & {\gray Lvl 2 + Report has specific lines of code that call the data and changes in the data produce traceable changes in the report} \\
     \midrule
      {\black Methods \& Code} & {\gray Key assumption are listed} & {\gray Methods are described in prose. Large amount of work is required to reproduce qualitatively similar estimates}  & {\gray Methods and described in prose, with detailed formulas, and code is provided as supplementary material} & {\gray Lvl 2 + All is in the same document where changes in the code affect the output automatically} \\
     \toprule
     \multicolumn{5}{ P{11.7cm} }{  \raggedleft {\red\small{ From TOP guidelines \citep{nosek2015promoting} v1.0.1}}}
   \end{tabular}}
\end{frame}







\end{document}

